\documentclass[12pt]{article} % use larger type; default would be 10pt

\usepackage[polish]{babel}
\usepackage[utf8]{inputenc}
\usepackage{t1enc}

\usepackage{amsfonts}
\usepackage{amsmath}

\usepackage{hyperref}
\usepackage{graphicx}

\addtolength{\textwidth}{3cm}
\addtolength{\textheight}{5.5cm}
\addtolength{\voffset}{-2.5cm}
\addtolength{\hoffset}{-1cm}

\renewcommand\baselinestretch{1.1}


\DeclareMathOperator{\Q}{\mathbb{Q}} 
\DeclareMathAlphabet{\mathpzc}{OT1}{pzc}{m}{it}

\def\R{\mathbb{R}} 
\def\N{\mathbb{N}}
\def\Z{\mathbb{Z}}

\def\CC{\mathcal{C}}
\def\FF{\mathcal{F}} 
\def\MM{\mathcal{M}} 
\def\NN{\mathcal{N}} 

\newcommand{\nnatural}{\mathbb{N}}
\newcommand{\rational}{\mathbb{Q}}
\newcommand{\real}{\mathbb{R}}
\newcommand{\integers}{\mathbb{Z}}
\newcommand{\complex}{\mathbb{C}}

\newcounter{zad}
\def\numzad{\smallskip\noindent\refstepcounter{zad}{\bf Zadanie \thezad.} }
\pagestyle{empty}

\date{}

\begin{document}

\bigskip
\centerline{\large{\bf{Zadania z kombinatoryki na Egzamin w roku 2023}}}
\bigskip

\numzad
W wagonie kolejowym jest $2 \cdot n$ ponumerowanych miejsc
w dwóch rzędach po $n$ miejsca.  Do pustego wagonu
weszło $k + m$ osób: $k$ pań i $m$ panów (zakładamy
że $k > m$).
Panie zajęły miejsca w jednym rzędzie a panowie usiedli 
w drugim. Oblicz na ile sposobów osoby te mogły zająć miejsca tak, aby vis-\`{a}-vis
każdego z panów siedziała pani.
Podaj wzór w przypadku kiedy:
\begin{enumerate}
\item
  Panie i panowie są {\bf rozróżnialni}
\item
  Panie i panowie {\bf nie są rozróżnialni} (tzn.
  odróżniamy panie od panów ale wszystkie panie
  i wszyscy panowie wyglądają tak samo i nie umiemy
  ich od siebie odróżnić).
\end{enumerate}

\smallskip 

Zastosuj znalezione wzory dla przypadku : $8$ miejsc po $4$
w każdym rzędzie, $3$ panie i $2$ panów.
W którym przypadku (rozróżnialni czy nie rozróżnialni panie i panowie)
będzie większa liczba kombinacji?

There are $2 \cdot n$ numbered seats in a train car 
in two rows of $n$ spaces. 
$k + m$ people entered: $k$ ladies and $m$ men (we assume
that $k > m$).
The ladies took their seats in a row and the gentlemen sat down
in second. Calculate in how many ways these people could take 
their seats so that vis-\`{a}-vis
each of the gentlemen was seated by a lady.
Please, consider two cases:
\begin{enumerate}
\item
   Ladies and gentlemen are {\bf distinguishable}
\item
   Ladies and gentlemen {\bf are not distinguishable} (i.e.
   we distinguish ladies from gentlemen but all ladies
   and all the gentlemen look the same and we can't
   distinguish them from each other).
\end{enumerate}

Next, apply the formulas you found to the case : $8$ places of $4$
in each row, $3$ ladies and $2$ gentlemen.
In which case (distinguishable or indistinguishable ladies and gentlemen)
will there be more combinations?

\numzad
Wykazać że wyznacznik macierzy:
(Prove that the determinant of the matrix:)
\begin{equation*}
\begin{pmatrix}
1 & -1 &  0 & \cdots & 0 \\
1 &  1 & -1 & \cdots & 0 \\
0 &  1  & 1 & \cdots & 0 \\
\ldots & \ldots & \ldots & \ldots & \ldots \\
\end{pmatrix}
\end{equation*}
(forms the Fibonacci sequence).
tworzy ciąg Fibonacciego. 

\numzad
Udowodnić kombinatorycznie że
(prove using combinatorial methods that):
\begin{enumerate}
\item
  $[p + q]_n = \sum_{k=0}^n {n \choose k} [p]_k [q]_{n-k}$
\item
  $[p + q]^n = \sum_{k=0}^n {n \choose k} [p]^k [q]^{n-k}$
\end{enumerate} 
 
 (przypomnienie (recall):
$[m]_n = m \cdot (m - 1) \ldots (m - n + 1)$ oraz (and)
$[m]^n = m \cdot (m + 1) \ldots (m + n - 1)$ 
).
Wskazówka: $[m]_n$ to wariacje bez powtórzeń czyli
wybór $n$ różnych elementów z $m$ obiektów, przy czym
liczy się kolejność ich wybierania.
(Hint: $[m]^n$ is a variation without repetitions).
$[m]^n$ to rozmieszczenie $n$ rozróżnialnych kulek
w $m$ rozróżnialnych pudełkach, przy czym w pudełku
może być wiele kulek i co więcej, liczy się
kolejność ich umieszczenia (czyli np. kulka A potem B jest
czymś innym niż najpierw kulka B a potem A).

($[m]^n$ is the distribution of $n$ distinguishable balls
in $m$ distinguishable boxes, where in a box
there can be many balls and, moreover, 
the order of their placement is important (i.e. the ball A then the B is
an other sequence than the ball B first and then the A).

\numzad
Wiemy że (it is known that) (recently result!)
$R(3, 3, 3) = 17$. Explain in detail what does it
exactly mean. Wyjaśnij szczegółowo co dokładnie
oznacza ten zapis.

\numzad 
  Podaj definicję liczb Catalana (jakąkolwiek)
  Na wykładzie była mowa KIEDY liczba Catalana $C_n$
  jest liczbą nieparzystą, proszę napisać to kryterium.
  Jaki jest wzór rekurencyjny na liczby Catalana?

  Write the definition of the Catalan numbers.
  Give the necessary and sufficient condition
  for a Catalan number to be an odd number.
  Could you possibly write a recurrence formula for
  the Catalan numbers?
\medskip 

\end{document}
