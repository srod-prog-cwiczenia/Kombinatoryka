\documentclass[12pt]{article} % use larger type; default would be 10pt

\usepackage[polish]{babel}
\usepackage[utf8]{inputenc}
\usepackage{t1enc}

\usepackage{amsfonts}
\usepackage{amsmath}

\usepackage{hyperref}
\usepackage{graphicx}

\addtolength{\textwidth}{3cm}
\addtolength{\textheight}{5.5cm}
\addtolength{\voffset}{-2.5cm}
\addtolength{\hoffset}{-1cm}

\renewcommand\baselinestretch{1.1}


\DeclareMathOperator{\Q}{\mathbb{Q}} 
\DeclareMathAlphabet{\mathpzc}{OT1}{pzc}{m}{it}

\def\R{\mathbb{R}} 
\def\N{\mathbb{N}}
\def\Z{\mathbb{Z}}

\def\CC{\mathcal{C}}
\def\FF{\mathcal{F}} 
\def\MM{\mathcal{M}} 
\def\NN{\mathcal{N}} 

\newcommand{\nnatural}{\mathbb{N}}
\newcommand{\rational}{\mathbb{Q}}
\newcommand{\real}{\mathbb{R}}
\newcommand{\integers}{\mathbb{Z}}
\newcommand{\complex}{\mathbb{C}}

\newcounter{zad}
\def\numzad{\smallskip\noindent\refstepcounter{zad}{\bf Zadanie \thezad.} }
\pagestyle{empty}

\date{}

\begin{document}

\bigskip
\centerline{\large{\bf{Zadania z kombinatoryki na Egzamin w roku 2024}}}
\bigskip
\numzad
Przed kasą ustawia się 70 osób mających monetę 1 PLN i 30 osób z monetą 2 PLN.
Koszt wejścia wynosi 1 PLN. Jakie jest prawdopodobieństwo że
kasa nie będzie miała kłopotu z wydaniem reszty.
Napisać ogólny wzór z którego korzystamy.

\medskip 

There are 70 people with a PLN 1 coin and 30 people with a PLN 2 coin standing in a queue in front of the cash register.
The entry cost is PLN 1. What is the probability that the cash register will have no problem giving the change. Write the general formula.

\numzad
Rozpisać dokładnie znaczenie równości $R(4,4) = 18$.

\medskip

Write the complete description (interpretation) of the equality
$R(4,4) = 18$.

\numzad
Używając kodów Pr\"ufera pokazać że prawdopodobieństwo że losowo wybrane
drzewo posiada wierzchołek $1$ jako liśc dąży do $\frac{1}{e}$ gdy
$n$ dąży do nieskończoności ($n$ to liczba wierzchołków drzewa).

Wskazówka: Najpierw trzeba wykazać że drzewo ma wierchołek $1$ jako liść
wtedy i tylko wtedy gdy w jego kodzie Pr\"ufera nie występuje $1$.

By the Pr\"ufer codes technique prove that the probability
that randomly chosen tree has the verticle $1$ as a leaf 
tends to the number $\frac{1}{e}$.

Hint: At first, prove that any tree has the verticle $1$ as a leaf
if and only if $1$ does not appear in the tree's Pr\"ufer code.
\end{document}
