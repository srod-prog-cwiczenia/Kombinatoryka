\documentclass[12pt]{article} % use larger type; default would be 10pt

\usepackage[polish]{babel}
\usepackage[utf8]{inputenc}
\usepackage{t1enc}

\usepackage{amsfonts}
\usepackage{amsmath}

\usepackage{hyperref}
\usepackage{graphicx}

\addtolength{\textwidth}{3cm}
\addtolength{\textheight}{5.5cm}
\addtolength{\voffset}{-2.5cm}
\addtolength{\hoffset}{-1cm}

\renewcommand\baselinestretch{1.1}


\DeclareMathOperator{\Q}{\mathbb{Q}} 
\DeclareMathAlphabet{\mathpzc}{OT1}{pzc}{m}{it}

\def\R{\mathbb{R}} 
\def\N{\mathbb{N}}
\def\Z{\mathbb{Z}}

\def\CC{\mathcal{C}}
\def\FF{\mathcal{F}} 
\def\MM{\mathcal{M}} 
\def\NN{\mathcal{N}} 

\newcommand{\nnatural}{\mathbb{N}}
\newcommand{\rational}{\mathbb{Q}}
\newcommand{\real}{\mathbb{R}}
\newcommand{\integers}{\mathbb{Z}}
\newcommand{\complex}{\mathbb{C}}

\newcounter{zad}
\def\numzad{\smallskip\noindent\refstepcounter{zad}{\bf Zadanie \thezad.} }
\pagestyle{empty}

\date{}

\begin{document}

\bigskip
\centerline{\large{\bf{Zadania z kombinatoryki na Egzamin w roku 2024}}}
\bigskip
\numzad
Przed kasą ustawia się 70 osób mających monetę 1 PLN i 30 osób z monetą 2 PLN.
Koszt wejścia wynosi 1 PLN. Jakie jest prawdopodobieństwo że
kasa nie będzie miała kłopotu z wydaniem reszty.
Napisać ogólny wzór z którego korzystamy.

\medskip 
\hrule
\medskip

There are 70 people with a PLN 1 coin and 30 people with a PLN 2 coin standing in a queue in front of the cash register.
The entry cost is PLN 1. What is the probability that the cash register will have no problem giving the change. Write the general formula.

\numzad
Rozpisać dokładnie znaczenie równości $R(4,4) = 18$ (liczba Ramseya).

\medskip 
\hrule
\medskip

Write the complete description (interpretation) of the equality
$R(4,4) = 18$ (The Ramsey number).

\numzad
Używając kodów Pr\"ufera pokazać że prawdopodobieństwo że losowo wybrane
drzewo posiada wierzchołek $1$ jako liśc dąży do $\frac{1}{e}$ gdy
$n$ dąży do nieskończoności ($n$ to liczba wierzchołków drzewa).

Wskazówka: Najpierw trzeba wykazać że drzewo ma wierchołek $1$ jako liść
wtedy i tylko wtedy gdy w jego kodzie Pr\"ufera nie występuje $1$.

\medskip 
\hrule
\medskip

By the Pr\"ufer codes technique prove that the probability
that randomly chosen tree has the verticle $1$ as a leaf 
tends to the number $\frac{1}{e}$.

Hint: At first, prove that any tree has the verticle $1$ as a leaf
if and only if $1$ does not appear in the tree's Pr\"ufer code.

\numzad
W reaktorze jądrowym znajdują się dwa rodzaje cząstek:
$\alpha$ i $\beta$. W każdej sekundzie cząstka $\alpha$
rozpada się na trzy cząstki typu $\beta$ i jednocześnie w tej
samej sekundzie każda cząstka typu $\beta$ rozpada się
na jedną cząstkę typu $\alpha$ i dwie cząstki typu $\beta$.
Na początku mamy tylko jedną cząstkę $\alpha$.
Ułożyć równanie rekurencyjne na sumaryczną liczbę cząstek w zależności
od czasu i wyliczyć ile będzie cząstek w piątej sekundzie reakcji.

Aby wyjaśnić opis (który notabene - zdaję sobie z tego sprawę :-) -
może nie być krystalicznie czysto precyzyjny) załączam poniżej
zestawienie dwóch początkowych sekund przebiegu:

\[
\alpha \rightarrow 3 \beta \rightarrow 3 \alpha + 6 \beta \rightarrow
6 \alpha + (3 \cdot 3 + 2 \cdot 6) \beta \ldots
\]

\medskip 
\hrule
\medskip

In a nuclear reactor we have two types of particles, namely:
$\alpha$ and $\beta$. Every second a $\alpha$ particle
splits into three $\beta$ particles and, simultaneously, in the
same second, every $\beta$ particle splits
into one $\alpha$ particle and two $\beta$ particles.
Suppose that at the beginning we have only one $\alpha$ particle.

  Find a recursive formula for the total number of particles
(depending on the number of seconds)  and calculate how many particles
there will be in the $5^{\mathrm{th}}$ second of the reaction.

  To clarify the description (which - OK, I know that :-) -
may not be crystal clear) I attached below
summary of the first two seconds of the reaction:

\[
\alpha \rightarrow 3 \beta \rightarrow 3 \alpha + 6 \beta \rightarrow
6 \alpha + (3 \cdot 3 + 2 \cdot 6) \beta \ldots
\]

\numzad
\begin{flushleft}{\bf Kwadraty grecko-łacinskie - zadanie teoretyczne z Wykładu}
\end{flushleft}
Snucie naszej krótkiej opowieści o kwadratach grecko-łacińskich
zaczniemy od tego zdarzenia: Słynny Euler podejrzewał
że nie istnieją kwadraty grecko-łacińskie dla liczb $n = 6, 10, 14, 18, \ldots$.
  Następnie niejaki MacNeish w roku 1921 (lub 1922 (nie znam precyzyjnej daty!))
opublikował rzekomy dowód tej hipotezy, niestety zawierał on błąd.
  To przypuszczalnie jedna z najbardziej znanych w historii matematyki
publikacji zawierających błędny dowód która - paradoksalnie - stanowiła
istotny wkład w rozwój kombinatoryki m.in. przez inspirację
dla poszukiwania poprawnego dowodu (lub jego obalenia). Palmę
pierwszeństwa odnośnie sławy bez wątpienia dzierży artykuł naukowy
z mniemanym ,,dowodem'' tzw. Hipotezy Czterech Barw opublikowanym przez Alfreda Kempego w roku 1879.

Kropkę nad ,,i'' postawiła dopiero praca  Bose, Shrikhande oraz Parker z roku
1959 w której autorzy obalają hipotezę Eulera.
Dziś wiemy że kwadraty grecko-łacińskie istnieją
dla wszystkich $n$ z wyjątkiem $n = 2$ oraz $n = 6$.

  Tu załączam znaleziony komputerowo słynny przykład kwadratu grecko łacińskiego
który oczywiście obala przypuszczenie Eulera:


\begin{tabular}{| c | c | c | c | c | c | c | c | c | c | }
\hline
 46  & 57 & 68 & 70 & 81 & 02 & 13 & 24 & 35 & 99  \\
\hline
 71  & 94 & 37 & 65 & 12 & 40 & 29 & 06 & 88 & 53  \\
\hline
 93  & 26 & 54 & 01 & 38 & 19 & 85 & 77 & 60 & 42  \\
\hline
 15 & 43 & 80 & 27 & 09 & 74 & 66 & 58 & 92 & 31  \\
\hline
 32 & 78 & 16 & 89 & 63 & 55 & 47 & 91 & 04 & 20  \\
\hline
 67  & 05 & 79 & 52 & 44 & 36 & 90 & 83 & 21 & 18  \\
\hline
 84  & 69 & 41 & 33 & 25 & 98 & 72 & 10 & 56 & 07  \\
\hline
  59 & 30 & 22 & 14 & 97 & 61 & 08 & 45 & 73 & 86  \\
\hline
  28 & 11 & 03 & 96 & 50 & 87 & 34 & 62 & 49 & 75  \\
\hline
  00 & 82 & 95 & 48 & 76 & 23 & 51 & 39 & 17 & 64  \\
\hline
\end{tabular}

{\bf Zadanie:}

Proszę podać definicję kwadratu grecko-łacińskiego i podać
jego przykład dla $n=5$.


\medskip 
\hrule
\medskip


\begin{flushleft}{\bf Graeco-Latin squares - a theory}
\end{flushleft}

  Let us begin our short story about Graeco-Latin squares
with this incident: Euler conjectured that there is no Graeco-Latin square 
for the numbers $n = 6, 10, 14, 18, \ldots$.
 In 1921 or perhaps in 1922 (I don't know the precise date) 
MacNeish published a purported proof of this conjecture. 
  This is probably one of the most known
mathematical publication with a flaw in the proof, which - paradoxically - gave
contribution to the development of combinatorics, by inspiration
to prove or disprove the conjecture. 
The article of Alfred Kempe with the "proof" of the The Four Color Theorem
published by Alfred Kempe in 1879 undoubtedly bears the palm
of the most famous artile with a flaw.
  While it is true that no such square of order six exists, such squares were found to exist for all other orders of the form $4k + 2$ by Bose, Shrikhande, and Parker in 1959, refuting the conjecture 
(and establishing unequivocally the invalidity of MacNeish's "proof").
  Today we know that Greaco-Latin squares constitute
for all $n$ except $n = 2$ and $n = 6$.

  Here I am attaching a computer-generated solution, an example of another Greek square
which of course refutes Euler's conjecture:


\begin{tabular}{| c | c | c | c | c | c | c | c | c | c | }
\hline
 46  & 57 & 68 & 70 & 81 & 02 & 13 & 24 & 35 & 99  \\
\hline
 71  & 94 & 37 & 65 & 12 & 40 & 29 & 06 & 88 & 53  \\
\hline
 93  & 26 & 54 & 01 & 38 & 19 & 85 & 77 & 60 & 42  \\
\hline
 15 & 43 & 80 & 27 & 09 & 74 & 66 & 58 & 92 & 31  \\
\hline
 32 & 78 & 16 & 89 & 63 & 55 & 47 & 91 & 04 & 20  \\
\hline
 67  & 05 & 79 & 52 & 44 & 36 & 90 & 83 & 21 & 18  \\
\hline
 84  & 69 & 41 & 33 & 25 & 98 & 72 & 10 & 56 & 07  \\
\hline
  59 & 30 & 22 & 14 & 97 & 61 & 08 & 45 & 73 & 86  \\
\hline
  28 & 11 & 03 & 96 & 50 & 87 & 34 & 62 & 49 & 75  \\
\hline
  00 & 82 & 95 & 48 & 76 & 23 & 51 & 39 & 17 & 64  \\
\hline
\end{tabular}

{\bf Question:}

Please write down the definition of a Graeco-Latin square
and give an example of such a square of size $5 \times 5$.

\end{document}
