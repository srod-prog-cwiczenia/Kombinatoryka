\documentclass[12pt]{article} % use larger type; default would be 10pt

\usepackage[polish]{babel}
\usepackage[utf8]{inputenc}
\usepackage{t1enc}

\usepackage{amsfonts}
\usepackage{amsmath}

\usepackage{hyperref}

\addtolength{\textwidth}{3cm}
\addtolength{\textheight}{5.5cm}
\addtolength{\voffset}{-2.5cm}
\addtolength{\hoffset}{-1cm}

\renewcommand\baselinestretch{1.1}


\DeclareMathOperator{\Q}{\mathbb{Q}} 
\DeclareMathAlphabet{\mathpzc}{OT1}{pzc}{m}{it}

\def\R{\mathbb{R}} 
\def\N{\mathbb{N}}
\def\Z{\mathbb{Z}}

\def\CC{\mathcal{C}}
\def\FF{\mathcal{F}} 
\def\MM{\mathcal{M}} 
\def\NN{\mathcal{N}} 

\newcommand{\rational}{\mathbb{Q}}
\newcommand{\real}{\mathbb{R}}
\newcommand{\integers}{\mathbb{Z}}
\newcommand{\complex}{\mathbb{C}}

\newcounter{zad}
\def\numzad{\smallskip\noindent\refstepcounter{zad}{\bf Zadanie \thezad.} }
\pagestyle{empty}

\date{}

\begin{document}

\bigskip
\centerline{\large{\bf{Zadania z kombinatoryki (seria nr 1)}}}
\bigskip

\numzad

Udowodnić że liczba sposobów którymi można rozsadzić $n$
spośród $m$ osób przy okrągłym stole jest równa
$\frac{[m]_n}{n}$ - utożsamiamy rozsadzenia różniące
sie jedynie cyklicznym przesunięciem osób wokół stołu.
Uwaga: $[m]_n = m \cdot (m - 1) \ldots (m - n + 1)$.

\numzad
Ile jest liczb naturalnych z przedziału 
$[1000, 10000]$ takich że:
\begin{enumerate}
\item
  wszystkie cyfry są nieparzyste;
\item
  wszystkie cyfry są parami różne.
\end{enumerate}
 
\numzad
 Udowodnić że:
\begin{enumerate}
\item
  $[p + q]_n = \sum_{k=0}^n {n \choose k} [p]_k [q]_{n-k}$
\item
  $[p + q]^n = \sum_{k=0}^n {n \choose k} [p]^k [q]^{n-k}$
\end{enumerate} 
 
 (przypomnienie:
$[m]_n = m \cdot (m - 1) \ldots (m - n + 1)$ oraz
$[m]^n = m \cdot (m + 1) \ldots (m + n - 1)$ 
 ).
 
\numzad
 Udowodnić (najlepiej kombinatorycznie, ale gdy nie
 będzie wychodzić, to analitycznie (=rachunkowo)) że
 ${n\choose 0} < {n\choose 1} < \ldots < {n\choose \lfloor\frac{n}{2}\rfloor} = 
 {n\choose \lceil\frac{n}{2}\rceil} > \ldots > {n\choose n}$.
 
\numzad
 Udowodnić że iloczyn dowolnych kolejnych $k$ liczb
 naturalnych jest podzielny przez $k!$ (wskazówka:
 skorzystać z faktu że każdy współczynnik dwumienny jest
 liczbą całkowitą, w szczególności ${n + k \choose n}$).
 
\numzad
 Ile  liczb naturalnych nie przekraczających $1000$
 nie dzieli się przez $3$, $7$ ani przez $11$?
 
\numzad
 Iloma sposobami można na szachownicy $8\times 8$ 
 rozmieścić $8$ wież (wieża porusza się w górę, w dół, w lewo lub w prawo
 o dowolną liczbę pól: 
 \url{https://pl.wikipedia.org/wiki/Wie%C5%BCa_(szachy)}
  ) tak aby każde dwie nie atakowały się wzajemnie
  i aby żadna nie leżała na wybranej przekątnej.
  
\numzad
  Znaleźć liczbę ciągów długości $2n$  takich że każda liczba
  $i \in {1,\ldots, n}$ występuje dokładnie dwa razy
  przy czym każde dwa kolejne wyrazy nie są równe.
  
  Wskazówka: Zastosować zasadę włączania - wyłączania
  \url{https://pl.wikipedia.org/wiki/Zasada_w%C5%82%C4%85cze%C5%84_i_wy%C5%82%C4%85cze%C5%84}
  
\numzad
  Ile jest podzbiorów $11$ elementowych zbioru z powtórzeniami
  gdzie: a - występuje 4 razy, b - 3 razy zaś c - 7 razy
  Wskazówka: Spróbować zastosować zasadę włączania - wyłączania.
  
  Przypomnienie:
{\bf Nieporządkiem} nazywamy permutację bez punktów stałych,
czyli taką 
bijekcję
$\sigma\colon {1,\ldots,n} \to {1,\ldots,n}$
że $\sigma(i) \not= i$.

Liczbę wszystkich nieporządków na zbiorze ${1,\ldots,n}$
oznaczamy przez $D_n$. 
Bardzo proszę o zapoznanie się z tym artykułem:
\url{https://pl.wikipedia.org/wiki/Nieporz%C4%85dek} 

Mamy:
  $D_n = n! \cdot \sum_{k = 0}^n \frac{(-1)^k}{k!}$.

\numzad
Udowodnić (jeśli się uda to najlepiej kombinatorycznie!)
zależności:
$D_{n+1} = (n + 1) D_n + (-1)^{n + 1}$.
$D_{n+1} = n (D_n + D_{n-1})$.

\numzad
Korzystając z jednej z możliwych interpretacji kombinatorycznych
liczb Fibonacciego (np że jest to liczba podziałów na kafelki
$2\times 1$ prostokąta $n\times 2$, albo
że jest to liczba rozmieszczeń lwów w $n$ klatkach tak aby
żadne dwie sąsiadujące klatki nie były ,,zajęte''), etc.,etc.
(warto zapoznać się - nawet pobieżnie - z tym obszernym artykułem:
\url{https://pl.wikipedia.org/wiki/Ci%C4%85g_Fibonacciego}
)
\begin{enumerate}
\item
  $F_{n+m} = F_n F_m + F_{n-1} F_{m-1}$;
\item
  $NWD(F_n,F_m) = F_{NWD(n+1,m+1) - 1}$,
  $m,n>1$.
  (chyba że odpowiednio dostosujemy (jak?) numerację liczb Fibonacciego,
  wtedy mamy nawet: $NWD(F_n,F_m) = F_{NWD(n,m)}$)
\end{enumerate}    

\numzad
Wykazać że wyznacznik macierzy:
\begin{equation*}
\begin{pmatrix}
1 & -1 &  0 & \cdots & 0 \\
1 &  1 & -1 & \cdots & 0 \\
0 &  1  & 1 & \cdots & 0 \\
\ldots & \ldots & \ldots & \ldots & \ldots \\
\end{pmatrix}
\end{equation*}
tworzy ciąg Fibonacciego. 
\end{document}
