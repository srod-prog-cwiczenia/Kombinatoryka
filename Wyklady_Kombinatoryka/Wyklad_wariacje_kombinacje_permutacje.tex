\newcommand{\real}{\mathbb{R}}
\newcommand{\nnatural}{\mathbb{N}}
\newcommand{\cE}{\mathcal{E}}
\newcommand{\cF}{\mathcal{F}}
\newcommand{\cI}{\mathcal{I}}
\newcommand{\cM}{{\mathcal{M}}}
\newcommand{\cN}{{\mathcal{N}}}
\newcommand{\cIMN}{\mathcal{I}_{\mathcal{N}, \mathcal{M}}}
\newcommand{\Fix}{\mathit{Fix}}
\newcommand{\stirlingii}{\genfrac{\{}{\}}{0pt}{}}

\documentclass{beamer} 
%\documentclass[a4paper, 11pt, xcolor=dvipsnames]{beamer} 
\usepackage[polish]{babel}
\usepackage[utf8]{inputenc}
\usepackage{t1enc}
\usepackage{amsmath}
\usepackage{graphics} 
\usepackage{fancybox} 
\usepackage{makecell}

%\usepackage{helvet}
%All font families are not available in every Beamer installation, but
%typically, at least some of the following families will be available:
%    serif          avant        bookman      chancery        charter
%     euler         helvet       mathtime       mathptm       mathptmx
%    newcent      palatino        pifont        utopia

\mode<presentation> {
\usetheme{Madrid} 
}
\usepackage{graphicx} % Allows including images
\usepackage{booktabs} % Allows the use of \toprule, \midrule and \bottomrule in tables

\usecolortheme[rgb={0.1,0.6,0.3}]{structure} 
%\usecolortheme[named=RubineRed]{structure} 
%\usecolortheme[named=CarnationPink]{structure}
%\usecolortheme[rgb={0.1,0.6,0.3}]{structure} 
\begin{document} 
%\setbeamercovered{transparent=15} %Ukryte (np przez uncover beda polwidoczne)
\setbeamercovered{dynamic}        %Ukryte (np przez uncover beda polwidoczne, im blizej tym mocniej)
%%% to ponizej tez nie dziala...
%\transglitter[direction=90]
%%% to nizej mialo zmienic kolor bibliografii lecz niestety nie zadzialalo.
%\setbeamercolor{bibliography item}{fg=black}
%\setbeamercolor*{bibliography entry title}{fg=black}
%-----------------------------------
\title[Wariacje, kombinacje, permutacje...(Variations, combinations, permutations)]{
  Wariacje, kombinacje, permutacje, liczby Stirlinga
  liczby podziałowe, liczby Bella, ,,pudełkologia'', etc.
  Variations, combinations, permutations, Stirling numbers,
  Bell numbers, ``carton-ology'', etc.
} 
%\subtitle{}
\author{} 
\date{\today}
%---------------------
\begin{frame} 
  \titlepage 
\end{frame} 
%---------------------
%\section[Outline of the talk]{}
%\frame{\tableofcontents}
%----------------------------------

\begin{frame}
  \uncover<1->{
    \begin{block}{Liczby podziałowe (The partition numbers)}
    \end{block}  
  }
\uncover<2->{
  {\bf Podział} liczby $n$ na $k$ składników to przedstawienie $n$
 w postaci sumy
$a_0 + \dots + a_{k-1} = n$,
gdzie $1 \leq a_0 \leq a_1 \leq \ldots a_{k-1} \leq n$
Liczbę podziałów $n$ na $k$ składników oznaczamy przez $P(n,k)$.
{\bf Ważne!} Nie liczy się kolejność i nie dopuszczamy tutaj zera.
}
\uncover<3->{
  A {\bf partition} of a given numer $n$ into $k$ parts
  is a representation of the number $n$ into the following sum:
$a_0 + \dots + a_{k-1} = n$,
where $1 \leq a_0 \leq a_1 \leq \ldots a_{k-1} \leq n$
The number of all partitions of $n$ into $k$ parts we denote by $P(n,k)$.
{\bf Please notice!} The order of these numbers is not important, also the
zero value is forbidden.
}
\uncover<4->{
Przykład (Example(: 
$P(7, 4) = 3$, bo mamy (since we have):
\begin{tabbing}
$7 = 1 + 1 + 1 + 4$ \\
$7 = 1 + 1 + 2 + 3$ \\
$7 = 1 + 2 + 2 + 2$ \\
\end{tabbing}
}
\end{frame}

\begin{frame}
  \uncover<1->{
    {\bf Liczbą podziałową} nazywamy liczbę wszystkich możliwych podziałów
    liczby $n$ na dodatnie składniki, oznaczamy ją przez $p(n)$.
  }
  \uncover<2->{
    {\bf The partition number} We consider {\it all} possible partitions of
    $n$ into nonzero parts. The number of all these partitions we denote by $p(n)$.
  }
  \uncover<3->{
    {\bf (Przykład) Example: }
    $p(6) = 11$, bo (since):
\begin{tabbing}
  $6 = 6$\\
  $6 = 5 + 1$\\
  $6 = 4 + 2$\\
  $6 = 4 + 1 + 1$\\
  $6 = 3 + 3$\\
  $6 = 3 + 2 + 1$\\
  $6 = 3 + 1 + 1 + 1$\\
  $6 = 2 + 2 + 2$\\
  $6 = 2 + 2 + 1 + 1$\\
  $6 = 2 + 1 + 1 + 1 + 1$\\
  $6 = 1 + 1 + 1 + 1 + 1 + 1$\\
\end{tabbing}  
  }
\end{frame}

\begin{frame}
\uncover<1->{
\begin{block}{,,Pudełkologia'' (Carton-ology :) ):}
\begin{table}
  \centering
    \begin{tabular}{| c | c | c |}
    \thead{n pudełek (boxes)\\ k obiektów (balls)}  & \thead{Pudełka rozróżnialne\\ (Distinguishable boxes)} & \thead{Pudełka NIErozróżnialne \\ (indistinguishable boxes)} \\
    \hline
    \thead{Obiekty rozróżnialne\\ (dist. balls)} & $n^k$  &  $\sum_{j = 1}^n \stirlingii{k}{j}$\\
    \hline
    \thead{Obiekty NIErozróżnialne\\ (indist.balls)}  & $\binom{n + k - 1}{k} = \binom{n + k - 1}{n - 1}$ &  $\sum_{j = 1}^n P(k, j) = P(n + k, n)$ \\
    \hline
    \end{tabular}
  \caption{Rozmieszczenia bez ograniczeń (without any restrictions)}
   \end{table}
\end{block}
}
\end{frame}

\begin{frame}
\uncover<1->{
\begin{block}{,,Pudełkologia - c.d.'' (Carton-ology - continued :) ):}
\begin{table}
  \centering
    \begin{tabular}{| c | c | c |}
    \hline
      \thead{n pudełek (boxes)\\ k obiektów (balls) \\ the assumption:\\ $k \leq n$} & \thead{Pudełka rozróżnialne\\ Distinguishable boxes} & \thead{Pudełka NIErozróżnialne\\ Indistinguishable boxes} \\
    \hline
    \thead{Obiekty rozróżnialne\\ dist.balls} & \thead{\small $[n]_k =$ \\ $n \cdot (n-1) \cdot \ldots \cdot (n - k + 1)$ \\
      $= \frac{n!}{(n-k)!}$}  &  $1$\\
    \hline
    \thead{Obiekty NIErozróżnialne\\ indist.balls}  & $\binom{n}{k}$ &  $1$ \\
    \hline
    \end{tabular}
  \caption{W każdym pudełku może być co najwyżej {\bf JEDEN} obiekt (WITH exclusion)}
   \end{table}
\end{block}
}
\end{frame}

\begin{frame}
\uncover<1->{
\begin{block}{,,Pudełkologia - c.d.'' (Carton-ology - continued :) ):}
\begin{table}
  \centering
    \begin{tabular}{| c | c | c |}
    \hline
      \thead{n pudełek (boxes)\\ k obiektów (balls) \\ the assumption:\\ $n \leq k$} & \thead{Pudełka rozróżnialne\\ Distinguishable boxes} & \thead{Pudełka NIErozróżnialne\\ Indistinguishable boxes} \\
    \hline
    \thead{Obiekty rozróżnialne\\ Dist.balls} & \thead{\small $s(k, n) =$ \\ $n! \stirlingii{k}{n}$ \\ (i.e. the number of all\\ surjections from a\\ $k-$ elem. set\\ onto an $n-$ elem. set) }  & \makecell{\small $\stirlingii{k}{n}$}\\
    \hline
    \thead{Obiekty NIErozróżnialne\\ Indist.balls} & $\binom{k - 1}{k - n} = \binom{k - 1}{n - 1}$ & $P(k, n)$ \\
    \hline
    \end{tabular}
  \caption{W każdym pudełku MUSI być co najmniej jeden obiekt}
   \end{table}
\end{block}
}
\end{frame}

\begin{frame}
\uncover<1->{
\begin{block}{,,Pudełkologia - c.d.'' (Carton-ology - continued :) ):}
\begin{table}
  \centering
    \begin{tabular}{| c | c | c |}
    \hline
      \thead{n pudełek (boxes)\\ k obiektów (balls)} & \thead{Pudełka rozróżnialne\\ Distinguishable boxes} & \thead{Pudełka NIErozróżnialne\\ Indistinguishable boxes} \\
    \hline
    \thead{Obiekty rozróżnialne\\ dist.balls} & \thead{\small $[n]^k =$ \\ $n \cdot (n+1) \cdot \ldots \cdot (n + k - 1)$ \\
      $= \binom{n + k - 1}{k} \cdot k!$}  &  ??? \\
    \hline
    \thead{Obiekty NIErozróżnialne\\ indist.balls}  & --- & --- \\
    \hline
    \end{tabular}
  \caption{W każdym pudełku liczy się kolejność umieszczania obiektów}
   \end{table}
\end{block}
}
\end{frame}

\begin{frame}
\begin{block}{Uzupełnienie o wzory (suplement):}
\end{block}
Liczba surjekcji ze zbioru $n$-elementowego $X$
na zbiór $m$-elementowy (przy czym oczywiście
zakładamy że $m \leq n$ wynosi
(A formula for the number of all surjections from the $n$-size
set $X$ onto the $m$-size set $Y$):
\[s(n,m) = \sum_{k = 0}^m (-1)^k \binom{m}{k} (m - k)^n
\]

We have also: \[
\stirlingii{n}{m} = \frac{s(n,m)}{m!}
\]
\end{frame}


\end{document}

