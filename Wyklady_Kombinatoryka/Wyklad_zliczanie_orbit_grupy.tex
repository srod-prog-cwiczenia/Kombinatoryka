\newcommand{\real}{\mathbb{R}}
\newcommand{\nnatural}{\mathbb{N}}
\newcommand{\cE}{\mathcal{E}}
\newcommand{\cF}{\mathcal{F}}
\newcommand{\cI}{\mathcal{I}}
\newcommand{\cM}{{\mathcal{M}}}
\newcommand{\cN}{{\mathcal{N}}}
\newcommand{\cIMN}{\mathcal{I}_{\mathcal{N}, \mathcal{M}}}
\newcommand{\Fix}{\mathit{Fix}}
\newcommand{\stirlingii}{\genfrac{\{}{\}}{0pt}{}}

\documentclass{beamer} 
%\documentclass[a4paper, 11pt, xcolor=dvipsnames]{beamer} 
\usepackage[polish]{babel}
\usepackage[utf8]{inputenc}
\usepackage{t1enc}
\usepackage{amsmath}
\usepackage{graphics} 
\usepackage{fancybox} 
\usepackage{makecell}

%\usepackage{helvet}
%All font families are not available in every Beamer installation, but
%typically, at least some of the following families will be available:
%    serif          avant        bookman      chancery        charter
%     euler         helvet       mathtime       mathptm       mathptmx
%    newcent      palatino        pifont        utopia

\mode<presentation> {
\usetheme{Madrid} 
}
\usepackage{graphicx} % Allows including images
\usepackage{booktabs} % Allows the use of \toprule, \midrule and \bottomrule in tables

%%% to co nizej to byl standardowy temat:
%\usecolortheme[rgb={0.1,0.6,0.3}]{structure}
%%% na razie taki :-)
\usecolortheme[named=violet]{structure}
%\usecolortheme[named=RubineRed]{structure} 
%\usecolortheme[named=CarnationPink]{structure}
\begin{document} 
%\setbeamercovered{transparent=15} %Ukryte (np przez uncover beda polwidoczne)
\setbeamercovered{dynamic}        %Ukryte (np przez uncover beda polwidoczne, im blizej tym mocniej)
%%% to ponizej tez nie dziala...
%\transglitter[direction=90]
%%% to nizej mialo zmienic kolor bibliografii lecz niestety nie zadzialalo.
%\setbeamercolor{bibliography item}{fg=black}
%\setbeamercolor*{bibliography entry title}{fg=black}
%-----------------------------------
\title[Zliczanie orbit grupy]{
  ...
} 
%\subtitle{}
\author{} 
\date{\today}
%---------------------
\begin{frame} 
  \titlepage 
\end{frame} 
%---------------------
%\section[Outline of the talk]{}
%\frame{\tableofcontents}
%----------------------------------
%\section{Introduction}
\begin{frame}
\uncover<1->{
\begin{block}{Zadania o wspólnym mianowniku}
\end{block}
}
\begin{itemize}

\item<1->{
Kolorujemy ,,szachownicę'' $2 \times 2$ za pomocą $n$ kolorów.
Ile istnieje ,,istotnie różnych'' sposobów jej pokolorowania?
(Let us color a ''chessboard'' $2 \times 2$ using $n$ colors.
How many strictly different colouring we can find?)
}
\item<2->{
Wybieramy ,,losowo'' permutację zbioru $n$ elementowego.
Jaka jest wartość oczekiwana liczby jej punktów stałych?
(Let us choose randomly a permutation $\sigma$ from $S_n$. Can we
find a formula for $E(|fix(\sigma)|)$?).
}
\item<3->{
Tworzymy ,,naszyjnik'' używając $3$ ,,pereł'' zielonych, $2$ czerwonych oraz jednej niebieskiej.
Na ile ,,istotnie różnych'' sposobów możemy to uczynić?
}
\end{itemize}
\end{frame}


\begin{frame}
  \begin{block}{Lemat Burnside'a}
    Jeśli grupa $G$ działa na zbiorze $X$ to
    liczba orbit $\Omega$ tego działania
    wyraża się wzorem:
    (Suppose that the group $G$ acts on the set $X$. Then
    the formula for counting the number of an orbits
    is here:)
    $|\Omega| = \frac{1}{|G|} \sum_{g\in G} |fix(g)|$
  \end{block}
  \begin{block}{Historical remarks}
    This lemma is sometimes called ''The lemma that is not Burnside's''
    or Cauchy ([1845], Frobenius [1887] lemma).
    \end{block}
\end{frame}

\begin{frame}
  W przypadku pokolorowania $n$ kolorami ,,szachownicy'' $2\times 2$
  dostajemy za pomocą Lematu Burnside'a wzór:
  $\frac{n^4 + 2 n^3 + 3 n^2 + 2 n }{8}$.

  {\tt Ćwiczenie: Obliczyć to}
  
\end{frame}

\begin{frame}
  \begin{block}{Pojęcie {\bf Wielomianu Cyklowego}}
    {\bf Definicja:} Wielomianem cyklowym działania grupy $G$
    na zbiorze $X$ nazywamy
    wielomian postaci:
    $P_G(z_1, \ldots, z_n) = \frac{1}{|G|} \sum_{g\in G} z_1^{c_1(g)} \cdots z_n^{c_n(g)}$
    
    
    \end{block}
  {\tt Ćwiczenie: Dlaczego wielomianem cyklowym dla ,,szachownicy $2\times 2$ jest:}
  {\tt Exercise: Explain why the cyclic polynomial for the ''chessboard'' of
    size $2\times 2$ is :}
  $\frac{z_1^4 + 3 z_2^2 + 2 z_4^1 + 2 z_1^2 z_2}{8}$

\end{frame}

\begin{frame}
  \begin{block}{Zadanie:}
    Rozważmy naszyjnik złożony z $6$ ,,pereł''
    i grupę jego rotacji (Uwaga: w kontekście tego zadania nie zaliczamy do elementów naszej grupy
    symetrii!).
    Znaleźć jego wielomian cyklowy.
  \end{block}
  Odpowiedź: $\frac{z_1^6 + 2 z_6 + z_2^3 + 2 z_3^2}{6}$.
\end{frame}

\begin{frame}
  Twierdzenie (Redfield [1927], P\'{o}lya [1937]). Niech $G$ działa na zbiorze
  $X$, $|X| = n$. Niech $Y$ zbiór skończony, $|Y| = m$.
  Wówczas współczynnik przy $t_1^{r_1} \cdots t_m^{r_m}$
  w wielomianie $P_G(t_1 + \ldots t_m, t_1^2 + \ldots + t_m^2, \ldots,
  t_1^n + \ldots t_m^n)$
  określa liczbę orbit w której kolor $i$-ty pojawia się dokładnie $r_i$ razy.
  \end{frame}
\end{document}

