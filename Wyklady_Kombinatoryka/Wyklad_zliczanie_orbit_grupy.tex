\newcommand{\real}{\mathbb{R}}
\newcommand{\nnatural}{\mathbb{N}}
\newcommand{\cE}{\mathcal{E}}
\newcommand{\cF}{\mathcal{F}}
\newcommand{\cI}{\mathcal{I}}
\newcommand{\cM}{{\mathcal{M}}}
\newcommand{\cN}{{\mathcal{N}}}
\newcommand{\cIMN}{\mathcal{I}_{\mathcal{N}, \mathcal{M}}}
\newcommand{\Fix}{\mathit{Fix}}
\newcommand{\stirlingii}{\genfrac{\{}{\}}{0pt}{}}

\documentclass{beamer} 
%\documentclass[a4paper, 11pt, xcolor=dvipsnames]{beamer} 
\usepackage[polish]{babel}
\usepackage[utf8]{inputenc}
\usepackage{t1enc}
\usepackage{amsmath}
\usepackage{graphics} 
\usepackage{fancybox} 
\usepackage{makecell}

%\usepackage{helvet}
%All font families are not available in every Beamer installation, but
%typically, at least some of the following families will be available:
%    serif          avant        bookman      chancery        charter
%     euler         helvet       mathtime       mathptm       mathptmx
%    newcent      palatino        pifont        utopia

\mode<presentation> {
\usetheme{Madrid} 
}
\usepackage{graphicx} % Allows including images
\usepackage{booktabs} % Allows the use of \toprule, \midrule and \bottomrule in tables

%%% to co nizej to byl standardowy temat:
%\usecolortheme[rgb={0.1,0.6,0.3}]{structure}
%%% na razie taki :-)
\usecolortheme[named=violet]{structure}
%\usecolortheme[named=RubineRed]{structure} 
%\usecolortheme[named=CarnationPink]{structure}
\begin{document} 
%\setbeamercovered{transparent=15} %Ukryte (np przez uncover beda polwidoczne)
\setbeamercovered{dynamic}        %Ukryte (np przez uncover beda polwidoczne, im blizej tym mocniej)
%%% to ponizej tez nie dziala...
%\transglitter[direction=90]
%%% to nizej mialo zmienic kolor bibliografii lecz niestety nie zadzialalo.
%\setbeamercolor{bibliography item}{fg=black}
%\setbeamercolor*{bibliography entry title}{fg=black}
%-----------------------------------
\title[Zliczanie orbit grupy]{
  ...
} 
%\subtitle{}
\author{} 
\date{\today}
%---------------------
\begin{frame} 
  \titlepage 
\end{frame} 
%---------------------
%\section[Outline of the talk]{}
%\frame{\tableofcontents}
%----------------------------------
%\section{Introduction}
\begin{frame}
\uncover<1->{
\begin{block}{Zadania o wspólnym mianowniku}
\end{block}
}
\begin{itemize}

\item<1->{
Kolorujemy ,,szachownicę'' $2 \times 2$ za pomocą $n$ kolorów.
Ile istnieje ,,istotnie różnych'' sposobów jej pokolorowania?
}
\item<2->{
Wybieramy ,,losowo'' permutację zbioru $n$ elementowego.
Jaka jest wartość oczekiwana liczby jej punktów stałych?
}
\item<3->{
Tworzymy ,,naszyjnik'' używając $3$ ,,pereł'' zielonych, $2$ czerwonych oraz jednej niebieskiej.
Na ile ,,istotnie różnych'' sposobów możemy to uczynić?
}
\end{itemize}
\end{frame}

\end{document}

