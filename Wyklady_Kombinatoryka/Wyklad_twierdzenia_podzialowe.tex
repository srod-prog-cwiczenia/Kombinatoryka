\newcommand{\real}{\mathbb{R}}
\newcommand{\nnatural}{\mathbb{N}}
\newcommand{\cE}{\mathcal{E}}
\newcommand{\cF}{\mathcal{F}}
\newcommand{\cI}{\mathcal{I}}
\newcommand{\cM}{{\mathcal{M}}}
\newcommand{\cN}{{\mathcal{N}}}
\newcommand{\cIMN}{\mathcal{I}_{\mathcal{N}, \mathcal{M}}}
\newcommand{\Fix}{\mathit{Fix}}
\newcommand{\stirlingii}{\genfrac{\{}{\}}{0pt}{}}

\documentclass{beamer} 
%\documentclass[a4paper, 11pt, xcolor=dvipsnames]{beamer} 
\usepackage[polish]{babel}
\usepackage[utf8]{inputenc}
\usepackage{t1enc}
\usepackage{amsmath}
\usepackage{graphics} 
\usepackage{fancybox} 
\usepackage{makecell}
\usepackage{listings}
\usepackage{mathtools}

%\usepackage{helvet}
%All font families are not available in every Beamer installation, but
%typically, at least some of the following families will be available:
%    serif          avant        bookman      chancery        charter
%     euler         helvet       mathtime       mathptm       mathptmx
%    newcent      palatino        pifont        utopia

\mode<presentation> {
\usetheme{Madrid} 
}
\usepackage{graphicx} % Allows including images
\usepackage{booktabs} % Allows the use of \toprule, \midrule and \bottomrule in tables

%%% to co nizej to byl standardowy temat:
%\usecolortheme[rgb={0.1,0.6,0.3}]{structure}
%%% na razie taki :-)

%%% to colotheme od Ani:
%\usecolortheme[rgb={0.5,0.3.0.9}]{structure}

\usecolortheme[named=violet]{structure}

%\usecolortheme[named=RubineRed]{structure} 
%\usecolortheme[named=CarnationPink]{structure}
\begin{document} 
%\setbeamercovered{transparent=15} %Ukryte (np przez uncover beda polwidoczne)
\setbeamercovered{dynamic}        %Ukryte (np przez uncover beda polwidoczne, im blizej tym mocniej)
%%% to ponizej tez nie dziala...
%\transglitter[direction=90]
%%% to nizej mialo zmienic kolor bibliografii lecz niestety nie zadzialalo.
%\setbeamercolor{bibliography item}{fg=black}
%\setbeamercolor*{bibliography entry title}{fg=black}
%-----------------------------------
\title[Twierdzenia podziałowe]{
  Partition theorems.
} 
%\subtitle{}
\author{} 
\date{\today}
%---------------------
\begin{frame} 
  \titlepage 
\end{frame} 
%---------------------
%\section[Outline of the talk]{}
%\frame{\tableofcontents}
%----------------------------------
%\section{Introduction}
\begin{frame}
\uncover<1->{
\begin{block}{The Ramsey Theorem}
  Notation:
  Suppose that $X$ is a set. Denote:
  $[X]^r = \lbrace A \subseteq X\colon |A| = r\rbrace$
  Suppose that $m$ is a natural number. Denote:
  $[m]^r = [\lbrace 0, 1, \ldots, m-1\rbrace]^r$.
\end{block}
}
\uncover<2->{
  \begin{block}{The formulation (Frank Ramsey, [1930]):}
    For each natural numbers $k, r, n$ there exists
    a number $m$ such that for any function
    $c\colon [m]^r \to \lbrace 1, \ldots k\rbrace$
    there exists a set $A \subseteq \lbrace 0, \ldots, m - 1\rbrace$
    of size $n$ such that the function $c$ has a one value only
    on the whole set $[A]^r$.
    \end{block}
}
\uncover<3->{
  The minimal $m$ with this property we called a {\bf Ramsey number}
  and we denote it by $R_k(n; r)$.
}
\uncover<4->{
  Historical remarks: The idea of this theorem
  firslty appeared in the article by Frank Ramsey in 1930:
  \beamertemplatebookbibitems
\begin{thebibliography}{100}{
\bibitem{R}
  {\sc F.P.Ramsey}, 
  {\em On a problem of formal logic}
  London Mathematical Society (1930) Vol s2-30, Issue 1, 264-286.
}
\end{thebibliography}
}
\end{frame}

\begin{frame}[fragile] %opcja fragile jest jak się okazuje tu nieodzowna bo używamy listingu programu (tak samo w przypadku verbatim, etc.)
  \uncover<1->{{\small
      \setbeamercolor{postit}{fg=black,bg=yellow}
      \begin{beamercolorbox}[sep=1em]{postit}
        The notion of {\bf quadratic residue} is crucial
        for the construction of an example of a graph which shows
        that $R(4,4) \geq 18$:
      \end{beamercolorbox}
  }}
  
\begin{lstlisting}[language=Python]
reszty_kwadratowe = set({})
for n in range(1, 17):
	reszty_kwadratowe.add((n * n) % 17)
print(reszty_kwadratowe)
\end{lstlisting}

\end{frame}

\begin{frame}
  \begin{block}{
    Notacja ,,strzałkowa'' Erd\"osa i Rado
    (An ``arrow notation of Erd\"os and Rado):}
  \end{block}
  \uncover<1->{
    $n \to (m_1, \ldots, m_k)^r$ if and only if
    for each partition: $[n]^r = A_1 \cup \ldots \cup A_k$
    there exists a ``color'' $1 \leq j \leq k$
    and a set $M \subseteq [n]$ of size $\geq m_j$ such
    that $[M]^r \subseteq A_j$.
  }
  \uncover<2->{
We abbreviate: $n \to (\underbrace{l,l,\ldots, l}_{\text{k $\times$}})^r$ as simply $n \to (l)^r_k$. 
  }
\end{frame}

\begin{frame}
  \begin{block}{
    We know only seven (sic! \includegraphics[height=5mm]{krzyczacy.jpeg})
    van der Waerden numbers.}
   \end{block}

\begin{table}
  \centering
  \begin{tabular}{| c | c c c |}
    \hline 
     length$\backslash$ colors: & 2 & 3 & 4 \\
    \hline
    3 & 9 & 27 & 76 \\
    \hline
    4 & 35 & 293 &  \\
    \hline
    5 & 178 &  &  \\
    \hline
    6 & 1132 & & \\
    \hline
    \end{tabular}
  \caption{All known van der Waerden numbers}
\end{table}
\end{frame}

\begin{frame}
  \begin{block}{The Hales-Jewett theorem (1963)}
    
  \end{block}

\uncover<1->{
\begin{block}{}
{\small
\setbeamercolor{postit}{fg=black,bg=cyan}
\begin{beamercolorbox}[sep=1em]{postit}
  Garść oznaczeń (A handful of notations):
\end{beamercolorbox}
}      
Suppose that $n$, $d$ are any natural numbers.
By $\prescript{d}{}{[n]}$ we denote the
collection of all finite sequences of a size
$k$ of elements from the set
$[n]$ (recall: $[n] = \lbrace 0, 1, \ldots, n-1\rbrace$).
Notice: This is an $d$ - dimensional {\bf cube} of size $n$ (elements).

A {\bf line} is a subcollection of $\prescript{d}{}{[n]}$ 
of the following form:
$\lbrace (a_j^{(1)}, a_j^{(2)}, \ldots, a_j^{(d)}) \colon j = 0, 1 ,\ldots n-1\rbrace$
where all the sequence $(a_j^{(i)})_{j=0,1\ldots, n-1}$
is either a constant number (from $[n]$) or an identity sequence (namely
$(0,1,\ldots,n-1)$ and, moreover, not all sequences are constant.
\end{block}
}
\end{frame} 

\begin{frame}
\begin{block}{}
{\small
\setbeamercolor{postit}{fg=black,bg=cyan}
\begin{beamercolorbox}[sep=1em]{postit}
  Theorem [Hales-Jewett, (1963)]:
\end{beamercolorbox}
}      
\end{block}
For each $k, n$ there exists a number $d$ such that if
we color the cube $\prescript{d}{}{[n]}$ by $k$ colors
then there exists at least one monochromatic line.
The lowest such a number $d$ we denote by
$HJ(n, k)$.
\end{frame}

\begin{frame}
\uncover<1->{
\begin{block}{}
{\small
\setbeamercolor{postit}{fg=black,bg=cyan}
\begin{beamercolorbox}[sep=1em]{postit}
  Niewiele wiadomo o konkretnych liczbach Halesa-Jewetta:
\end{beamercolorbox}
}      
\end{block}
}
,,Trywialne'' wartości to np. $HJ(2, k) = k$.
            [N.Hindman, E.Tressler, 2014]: $HJ(3,2) = 4$.
            A już np wiadomo tylko że $HJ(4, 2) \leq 10^{11}$.
            
\end{frame}

\begin{frame}
\uncover<1->{
\begin{block}{}
{\small
\setbeamercolor{postit}{fg=black,bg=cyan}
\begin{beamercolorbox}[sep=1em]{postit}
  Twierdzenie Schura:
\end{beamercolorbox}
}      
\end{block}
}
\uncover<2->{
For each number of colors: $k$ one can find such a great natural number
$N$ such that if we color all natural numbers from
$1$ to $N$ by $k$ colors, then there exists a monochromatic
triple of the form: $n, m, n + m$ (all $\leq N$ of course).
The lowest $N$ with this property we call {\bf The Schur's numbers}
and we denote by $S(k)$.
}
\uncover<3->{
  We should be ashamed of our lack of knowledge of
  the Schur's number, since we know only that:
}
\uncover<4->{
\begin{table}
  \centering
  \begin{tabular}{| c | c |}
    \hline 
      k & S(k) \\
    \hline
      1 & 2 \\
    \hline
      2 & 5  \\
    \hline
      3 & 14 \\
    \hline
      4 & 45 \\
      \hline
      5 & 161 \\
      \hline
    \end{tabular}
  \caption{All known Schur's numbers}
\end{table}
}
\end{frame}

\begin{frame}
\uncover<1->{
  \begin{block}{Zależności:}
    {\small
      \setbeamercolor{postit}{fg=black,bg=cyan}
      \begin{beamercolorbox}[sep=1em]{postit}
        Mamy następujące wynikania:
      \end{beamercolorbox}
    }
    Hales-Jewett's Theorem $\implies$ Gallai Theorem $\implies$ Van der Waerden Theorem
    
    Ramsey Theorem $\implies$ Schur's Theorem
  \end{block}
}
\end{frame}

\end{document}

