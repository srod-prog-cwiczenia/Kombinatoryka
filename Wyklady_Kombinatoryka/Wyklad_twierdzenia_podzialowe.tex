\newcommand{\real}{\mathbb{R}}
\newcommand{\nnatural}{\mathbb{N}}
\newcommand{\cE}{\mathcal{E}}
\newcommand{\cF}{\mathcal{F}}
\newcommand{\cI}{\mathcal{I}}
\newcommand{\cM}{{\mathcal{M}}}
\newcommand{\cN}{{\mathcal{N}}}
\newcommand{\cIMN}{\mathcal{I}_{\mathcal{N}, \mathcal{M}}}
\newcommand{\Fix}{\mathit{Fix}}
\newcommand{\stirlingii}{\genfrac{\{}{\}}{0pt}{}}

\documentclass{beamer} 
%\documentclass[a4paper, 11pt, xcolor=dvipsnames]{beamer} 
\usepackage[polish]{babel}
\usepackage[utf8]{inputenc}
\usepackage{t1enc}
\usepackage{amsmath}
\usepackage{graphics} 
\usepackage{fancybox} 
\usepackage{makecell}
\usepackage{listings}

%\usepackage{helvet}
%All font families are not available in every Beamer installation, but
%typically, at least some of the following families will be available:
%    serif          avant        bookman      chancery        charter
%     euler         helvet       mathtime       mathptm       mathptmx
%    newcent      palatino        pifont        utopia

\mode<presentation> {
\usetheme{Madrid} 
}
\usepackage{graphicx} % Allows including images
\usepackage{booktabs} % Allows the use of \toprule, \midrule and \bottomrule in tables

%%% to co nizej to byl standardowy temat:
%\usecolortheme[rgb={0.1,0.6,0.3}]{structure}
%%% na razie taki :-)
\usecolortheme[named=violet]{structure}
%\usecolortheme[named=RubineRed]{structure} 
%\usecolortheme[named=CarnationPink]{structure}
\begin{document} 
%\setbeamercovered{transparent=15} %Ukryte (np przez uncover beda polwidoczne)
\setbeamercovered{dynamic}        %Ukryte (np przez uncover beda polwidoczne, im blizej tym mocniej)
%%% to ponizej tez nie dziala...
%\transglitter[direction=90]
%%% to nizej mialo zmienic kolor bibliografii lecz niestety nie zadzialalo.
%\setbeamercolor{bibliography item}{fg=black}
%\setbeamercolor*{bibliography entry title}{fg=black}
%-----------------------------------
\title[Twierdzenia podziałowe]{
  Partition theorems.
} 
%\subtitle{}
\author{} 
\date{\today}
%---------------------
\begin{frame} 
  \titlepage 
\end{frame} 
%---------------------
%\section[Outline of the talk]{}
%\frame{\tableofcontents}
%----------------------------------
%\section{Introduction}
\begin{frame}
\uncover<1->{
\begin{block}{The Ramsey Theorem}
  Notation:
  Suppose that $X$ is a set. Denote:
  $[X]^r = \lbrace A \subseteq X\colon |A| = r\rbrace$
  Suppose that $m$ is a natural number. Denote:
  $[m]^r = [\lbrace 0, 1, \ldots, m-1\rbrace]^r$.
\end{block}
}
\uncover<2->{
  \begin{block}{The formulation (Frank Ramsey, [1930]):}
    For each natural numbers $k, r, n$ there exists
    a number $m$ such that for any function
    $c\colon [m]^r \to \lbrace 1, \ldots k\rbrace$
    there exists a set $A \subseteq \lbrace 0, \ldots, m - 1\rbrace$
    of size $n$ such that the function $c$ has a one value only
    on the whole set $[A]^r$.
    \end{block}
}
\uncover<3->{
  The minimal $m$ with this property we called a {\bf Ramsey number}
  and we denote it by $R_k(n; r)$.
}
\uncover<4->{
  Historical remarks: The idea of this theorem
  firslty appeared in the article by Frank Ramsey in 1930:
  \beamertemplatebookbibitems
\begin{thebibliography}{100}{
\bibitem{R}
  {\sc F.P.Ramsey}, 
  {\em On a problem of formal logic}
  London Mathematical Society (1930) Vol s2-30, Issue 1, 264-286.
}
\end{thebibliography}
}
\end{frame}

\begin{frame}[fragile] %opcja fragile jest jak się okazuje tu nieodzowna bo używamy listingu programu (tak samo w przypadku verbatim, etc.)
\begin{lstlisting}[language=Python]
reszty_kwadratowe = set({})
for n in range(1, 17):
	reszty_kwadratowe.add((n * n) % 17)
print(reszty_kwadratowe)
\end{lstlisting}
\end{frame}

\end{document}

