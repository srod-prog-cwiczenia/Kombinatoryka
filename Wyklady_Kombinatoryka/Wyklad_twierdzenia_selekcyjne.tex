\newcommand{\real}{\mathbb{R}}
\newcommand{\nnatural}{\mathbb{N}}
\newcommand{\cE}{\mathcal{E}}
\newcommand{\cF}{\mathcal{F}}
\newcommand{\cI}{\mathcal{I}}
\newcommand{\cM}{{\mathcal{M}}}
\newcommand{\cN}{{\mathcal{N}}}
\newcommand{\cIMN}{\mathcal{I}_{\mathcal{N}, \mathcal{M}}}
\newcommand{\Fix}{\mathit{Fix}}

\documentclass{beamer} 
%\documentclass[a4paper, 11pt, xcolor=dvipsnames]{beamer} 
\usepackage[polish]{babel}
\usepackage[utf8]{inputenc}
\usepackage{t1enc}
\usepackage{amsmath}
\usepackage{graphics} 
\usepackage{fancybox} 

%\usepackage{helvet}
%All font families are not available in every Beamer installation, but
%typically, at least some of the following families will be available:
%    serif          avant        bookman      chancery        charter
%     euler         helvet       mathtime       mathptm       mathptmx
%    newcent      palatino        pifont        utopia

\mode<presentation> {
\usetheme{Madrid} 
}
\usepackage{graphicx} % Allows including images
\usepackage{booktabs} % Allows the use of \toprule, \midrule and \bottomrule in tables

\usecolortheme[rgb={0.1,0.6,0.3}]{structure} 
%\usecolortheme[named=RubineRed]{structure} 
%\usecolortheme[named=CarnationPink]{structure}
%\usecolortheme[rgb={0.1,0.6,0.3}]{structure} 
\begin{document} 
%\setbeamercovered{transparent=15} %Ukryte (np przez uncover beda polwidoczne)
\setbeamercovered{dynamic}        %Ukryte (np przez uncover beda polwidoczne, im blizej tym mocniej)
%%% to ponizej tez nie dziala...
%\transglitter[direction=90]
%%% to nizej mialo zmienic kolor bibliografii lecz niestety nie zadzialalo.
%\setbeamercolor{bibliography item}{fg=black}
%\setbeamercolor*{bibliography entry title}{fg=black}
%-----------------------------------
\title[Selektory i transwersale]{
  Selektory i transwersale, twierdzenia minimaksowe,
  Selectors and transversals, min-max theorems.
} 
%\subtitle{}
\author{} 
\date{\today}
%---------------------
\begin{frame} 
  \titlepage 
\end{frame} 
%---------------------
%\section[Outline of the talk]{}
%\frame{\tableofcontents}
%----------------------------------
%\section{Introduction}
\begin{frame}
\uncover<1->{
\begin{block}{Selektor versus transwersala - Selector vs transversal}
  Selektor - wybór po jednym punkcie z rodziny rozłącznych zbiorów, formalnie:
  A selector - choose one by one points from a given family of pairwise
  disjoint sets, more formally:
\end{block}
}
\uncover<2->{
\begin{block}{Selektor - selector} 
  Rodzina $(A_i)$ ma być złożona ze zbiorów parami rozłącznych.
  $A_i \cap A_j = \emptyset$ dla $i \not= j$,
  $|S \cap A_i| = 1$ dla każdego $i$.

  $(A_i)$ is a family of pairwise disjoint sets:
  $A_i \cap A_j = \emptyset$ for $i \not= j$,
  $|S \cap A_i| = 1$ for each $i$.
  
\end{block}
}
\uncover<3->{
\begin{block}{Transwersala - Transversal}
  Rodzina $(A_i)$ dowolna.
  Any family $(A_i)$.
Transwersala to wybór $a_i \in A_i$ taki że
$a_i \not= a_j$ dla $i \not= j$.
Transversal is a sequence $(a_i)$ of
pairwise different elements such
that $a_i \in A_i$.
\end{block}
}
\end{frame}

\begin{frame}
\uncover<1->{
Twierdzenie (Halla, 1935): Rodzina $(A_i\colon i\in I)$ ma transwersalę
wtedy i tylko wtedy gdy spełnia tak zwany 
{\bf warunek Halla}, czyli gdy

Theorem (Philip Hall, 1935): A family $(A_i \colon i \in I)$ has a
transveral iff it fulfills the following condition:
\[
\forall_{\emptyset\not=J\subseteq I} |\bigcup_{j\in J} A_j| \geq |J|
\]

}
\end{frame}

\begin{frame}{Garść uwag}{A few remarks}
\uncover<1->{
Dowód w jedną stronę (którą? :) ) jest oczywisty,
w drugą stronę jest bardziej złożony.
}
\uncover<2->{
Twierdzenie to ma mnóstwo równoważnych wersji,
grafową, twierdzenie o łańcuchach i antyłańcuchach
(=twierdzenie Dilwortha, etc.)
}
\uncover<3->{
Jak szybko można znaleźć transwersalę?
}
\uncover<4->{
Istnieje szybki (w miarę) algorytm do jej znajdowania
}
\end{frame}

\begin{frame}
\uncover<1->{
  Twierdzenie o wspólnych selektorach:
Jeżeli $A_1,\ldots A_n$ i $B_1,\ldots, B_n$ są
partycjami zbioru $X$ na zbiory tej samej mocy
to istnieje wspólny selektor, więcej nawet:
można podzielić $X$ na rozłączne wspólne selektory.
}
\uncover<2->{
  Zastosowanie: Jeśli rozdamy $52$ karty na $13$
  ,,lew'' (termin brydżowy, to grupa $4$ kart)
  to z każdej lewy można wybrać po jednej karcie w taki
  sposób że układ wybranych kart będzie tworzył
  wzór: AKQJ10 98765432 (kolory kart nie mają znaczenia).
  Dla pozostałych na stole kart można to powtórzyć, etc.
}
\end{frame}


\begin{frame}
\uncover<1->{
  Theorem about common selectors:
  Suppose that $A_1,\ldots A_n$ i $B_1,\ldots, B_n$ are
  two partitions of the set $X$
  into sets of the same size, then there
  exists a common selector, moreover,
  we can divide the set $X$ into pariwise disjoint
common selectors.
}

\uncover<2->{
  An application: Suppose that we deal the $52$
  cards into $13$ groups (calles ``tricks'') of $4$ cards,
  then we can take from each trick one card only
  to obtain the pattern:
  AKQJ10 98765432 (colours are irrelevant).
  For the rest of the cards we can repeat this procedure, et
  cetera.
}
\end{frame}

\begin{frame}
  \uncover<1->{
  \begin{block}
    The Dilworth's Theorem [1950].
  \end{block}
  }
  \uncover<2->{
  $\langle \mathbb{P}, \leq\rangle$
  }
  \uncover<3->{
  \begin{block}{The Dilworth's Theorem}  
    Maksymalny rozmiar {\bf antyłańcucha} = minimalnej liczbie {\bf łańcuchów}
    które pokrywają zbiór $\mathbb{P}$.

    Max size of an {\bf antichain} = min size of family of {\bf chains}
    which covers the set $\mathbb{P}$.

  \end{block}  
  }
  \uncover<4->{
    \begin{block}{The dual Dilworth's Theorem
        (sometimes called The Mirsky's Theorem}
    Maksymalny rozmiar {\bf łańcucha} = minimalnej liczbie {\bf antyłańcuchów}
    które pokrywają zbiór $\mathbb{P}$.

    Max size of a {\bf chain} = min size of family of {\bf antichains}
    which covers the set $\mathbb{P}$.

    \end{block}
  }
  \end{frame}

\begin{frame}
\uncover<1->{
  \begin{block}{Definicja}
{\small
\setbeamercolor{postit}{fg=black,bg=cyan}
\begin{beamercolorbox}[sep=1em]{postit}
Rodzina Spernera - The Sperner family
\end{beamercolorbox}
}      
To antyłańcuch w zbiorze $\langle P(X), \subseteq \rangle$.
It is an antichain in the set $\langle P(X), \subseteq \rangle$.
\end{block}
}
\uncover<2->{
  \begin{block}{Twierdzenie Spernera - Sperner's theorem [1928][Emanuel Sperner]}
    Each Sperner family has the size $\leq {n \choose \lfloor \frac{n}{2}
      \rfloor}$.
    \end{block}
}
\uncover<3->{
  Nie należy mylić tego twierdzenia z tak zwanym Lematem Spernera
  Don't confuse this theorem with so called ``Sperner's Lemma''.
}
\end{frame}

\begin{frame}
\uncover<1->{
  \begin{block}{A proof?}
    Sketch:
    At first, we will prove the
    {\bf Lubell - Yamamoto - Meshalkin inequality}:
    \[
\sum_{k=0}^n \frac{a_k}{{n \choose k}} \leq 1,
\]
where $a_k = |\lbrace A \in \mathcal{A}\colon |A| = k\rbrace$.
    \end{block}
}
\end{frame}
\end{document}

