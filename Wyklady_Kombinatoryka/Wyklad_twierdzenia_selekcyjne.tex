\newcommand{\real}{\mathbb{R}}
\newcommand{\nnatural}{\mathbb{N}}
\newcommand{\cE}{\mathcal{E}}
\newcommand{\cF}{\mathcal{F}}
\newcommand{\cI}{\mathcal{I}}
\newcommand{\cM}{{\mathcal{M}}}
\newcommand{\cN}{{\mathcal{N}}}
\newcommand{\cIMN}{\mathcal{I}_{\mathcal{N}, \mathcal{M}}}
\newcommand{\Fix}{\mathit{Fix}}

\documentclass{beamer} 
%\documentclass[a4paper, 11pt, xcolor=dvipsnames]{beamer} 
\usepackage[polish]{babel}
\usepackage[utf8]{inputenc}
\usepackage{t1enc}
\usepackage{amsmath}
\usepackage{graphics} 
\usepackage{fancybox} 

%\usepackage{helvet}
%All font families are not available in every Beamer installation, but
%typically, at least some of the following families will be available:
%    serif          avant        bookman      chancery        charter
%     euler         helvet       mathtime       mathptm       mathptmx
%    newcent      palatino        pifont        utopia

\mode<presentation> {
\usetheme{Madrid} 
}
\usepackage{graphicx} % Allows including images
\usepackage{booktabs} % Allows the use of \toprule, \midrule and \bottomrule in tables

\usecolortheme[rgb={0.1,0.6,0.3}]{structure} 
%\usecolortheme[named=RubineRed]{structure} 
%\usecolortheme[named=CarnationPink]{structure}
%\usecolortheme[rgb={0.1,0.6,0.3}]{structure} 
\begin{document} 
%\setbeamercovered{transparent=15} %Ukryte (np przez uncover beda polwidoczne)
\setbeamercovered{dynamic}        %Ukryte (np przez uncover beda polwidoczne, im blizej tym mocniej)
%%% to ponizej tez nie dziala...
%\transglitter[direction=90]
%%% to nizej mialo zmienic kolor bibliografii lecz niestety nie zadzialalo.
%\setbeamercolor{bibliography item}{fg=black}
%\setbeamercolor*{bibliography entry title}{fg=black}
%-----------------------------------
\title[Selektory i transwersale]{
Selektory i transwersale
} 
%\subtitle{}
\author{} 
\date{\today}
%---------------------
\begin{frame} 
  \titlepage 
\end{frame} 
%---------------------
%\section[Outline of the talk]{}
%\frame{\tableofcontents}
%----------------------------------
%\section{Introduction}
\begin{frame}
\uncover<1->{
\begin{block}{Selektor versus transwersala}
Selektor - wybór po jednym punkcie z rodziny rozłącznych zbiorów, formalnie:
\end{block}
}
\uncover<2->{
\begin{block}{Selektor} 
  Rodzina $(A_i)$ ma być złożona ze zbiorów parami rozłącznych.
  $A_i \cap A_j = \emptyset$ dla $i \not= j$,
  $|S \cap A_i| = 1$ dla każdego $i$.
\end{block}
}
\uncover<3->{
\begin{block}{Transwersala}
Rodzina $(A_i)$ dowolna.
Transwersala to wybór $a_i \in A_i$ taki że
$a_i \not= a_j$ dla $i \not= j$.
\end{block}
}
\end{frame}

\begin{frame}
\uncover<1->{
Twierdzenie (Halla, 1935): Rodzina $(A_i\colon i\in I)$ ma transwersalę
wtedy i tylko wtedy gdy spełnia tak zwany 
{\bf warunek Halla}, czyli gdy
\[
\forall_{\emptyset\not=J\subseteq I} |\bigcup_{j\in J} A_j| \geq |J|
\]

}
\end{frame}

\begin{frame}{Garść uwag}
\uncover<1->{
Dowód w jedną stronę (którą? :) ) jest oczywisty,
w drugą stronę jest bardziej złożony.
}
\uncover<2->{
Twierdzenie to ma mnóstwo równoważnych wersji,
grafową, twierdzenie o łańcuchach i antyłańcuchach
(=twierdzenie Dilwortha, etc.)
}
\uncover<3->{
Jak szybko można znaleźć transwersalę?
}
\uncover<4->{
Istnieje szybki (w miarę) algorytm do jej znajdowania
}
\end{frame}

\begin{frame}
Twierdzenie o wspólnych selektorach:
Jeżeli $A_1,\ldots A_n$ i $B_1,\ldots, B_n$ są
partycjami zbioru $X$ na zbiory tej samej mocy
to istnieje wspólny selektor, więcej nawet:
można podzielić $X$ na rozłączne wspólne selektory.
\end{frame}

\end{document}

